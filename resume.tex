% https://towardsdatascience.com/create-your-professional-educational-resume-using-latex-7bc371f201e3
\documentclass{resume} % class file: resume.cls 
\usepackage[left=0.75in,top=0.6in,right=0.75in,bottom=0.6in]{geometry}% left=<margin> 
\newcommand{\tab}[1]{\hspace{.2667\textwidth}\rlap{#1}}% newcommand -> https://qiita.com/zr_tex8r/items/5067307890d36c0e4882
\newcommand{\itab}[1]{\hspace{0em}\rlap{#1}}
\name{Rintaro Sato} % Your name
\address{Email \\ 310rnomeado@gmail.com}
\address{GitHub \\ https://github.com/rintarooo}
\address{LinkedIn \\ https://www.linkedin.com/in/rintarooo} 

\jobtitle{1st-year Master's student in CS}
\jobtitleA{Key Terms: Computer Vision, Reinforcement Learning, Combinatorial Optimization}
% \jobtitle{Master's student in Computer Science, Computer Vision, Reinforcement Learning, Combinatorial Optimization}% https://knews.vip/postcrawl/stack/view?site=te&key=13499794&alias=kono-tenpure-to-no-namae-no-shita-ni-midashi-o-tsuikashiyou-to-shiteimasu-ga-tsuikasuru-hoho-ga-wakarimasen

\begin{document}

\begin{rSection}{Education}

% \bf represents Bold, while \em represents italics and \hfill command is used to provide appropriate formatting space between the right and side of the page
{\bf Tsukuba University} \hfill {Japan} 
\\ M.E. in Computer Science.\hfill {\em Apr 2021 - present}
\\ Overall GPA: -

{\bf Kyushu University} \hfill {Japan} 
\\ B.E. in Mechanical Engineering.\hfill {\em Apr 2016 - Mar 2021}
\\ Thesis Theme: Solving The-Multi Depot Vehicle Routing Problem using Deep Reinforcement Learning
\\ Overall GPA: 3.31/4.00


\end{rSection}

% \begin{rSection}{Career Objective}
%  To work for an organization which provides me the opportunity to improve my skills and knowledge to grow along with the organization objective.
% \end{rSection}

% \begin{rSection}{Projects}
% {\bf GitHub Notifier}
% \\This project aims at providing real time information of events from GitHub and notify you accordingly. The project is in ready-to-deployment stage on a demo server as a cron-job. The notification engine  used for real time tracking is completely based on the python implementation assembled in an Android App with firebase cloud support.

% {\bf GitHub Notifier}
% \\This project aims at providing real time information of events from GitHub and notify you accordingly. The project is in ready-to-deployment stage on a demo server as a cron-job. The notification engine  used for real time tracking is completely based on the python implementation assembled in an Android App with firebase cloud support.

% \end{rSection}

\begin{rSection}{Technical Strengths}

\begin{tabular}{ @{} >{\bfseries}l @{\hspace{6ex}} l }
Coding \ & C / C++ / Python3 / MATLAB / PostgreSQL /  Shell Script \\
DL Framework & PyTorch / TensorFlow / Keras \\
Library & OpenCV / PCL / OpenAI Gym / Scikit-Learn\\
3D Modeling & Crep Parametric\\
Other Tools & Linux(Ubuntu) / Docker / Git / Vim / LaTeX / CMake\\
Launguage & fluent in English(TOEFL iBT: 78/120 / TOEIC: 925/990) / Japanese(native)
\end{tabular}
\end{rSection}
% 
% 
% \newpage
\begin{rSection}{Internship}
\begin{rSubsection}{Fusic Co., Ltd.}{part-time, Aug - Sep 2020}{Research Engineer(Computer Vision) in Machine Learning Team}{Japan}
 \item Perform color space conversion to RGB images with LAB / HSV and other image processing.
 \item Build person's image classification model using Deep Metric Learning / SVM / Decision Tree.
\end{rSubsection}
\begin{rSubsection}{Panasonic R\&D Center Singapore}{full-time, Dec 2019 - Mar 2020}{Research Engineer(Computer Vision) in Learning\&Vision Team}{Singapore}
 \item Work on projects in the following fields, Video Analysis, Action Recognition(event scene detection in Basketball videos, surveillance videos of prison).
 \item Build classification models(e.g., 3D-CNN, Two-Stream model with Optical Flow) on public dataset(e.g., UCF101, UCF-Crime) and our own dataset.
 \item Research SotA(State-of-the-Art) model architecture and training method, tune hyperparameter, evaluate pre-trained model performance.
\end{rSubsection}

\end{rSection}
% 
\newpage
% \begin{rSection}{Academic Achievements} 
% \item Project 'XYZ' won Best Project under Environmental Solver category under AICTE, Government of India
% \item Recieved Scholarship For Higher Education(She) Component Under Inspire Scheme worth INR 4,00,000
% \end{rSection}

\begin{rSection}{Extra-Cirrucular} 
\item {\bf TU Bergakademie Freiberg}\hfill {Germany}\\Internship Program\hfill {\em Sep - Nov 2019}\\- Conduct TGA(thermogravimetric analysis) experiments\\- Analyze results of experimental data with Python, pandas.
\item {\bf University of Illinois at Urbana-Champaign}\hfill {America}\\Research Exchange Program\hfill {\em Feb - Mar 2019}\\- Analyze images and videos for the oscillation of the water droplets on hydrophobic surfaces using MATLAB and high speed camera.
\item {\bf SEEDS}\hfill {Iceland}\\International Volunteer\hfill {\em Aug - Sep 2018}\\- Work on nature and environmental oriented tasks including maintaining hiking paths, building stairs and collecting pine cones that are used to plant again in the forest in the North of Iceland.
\item {\bf University of Cambridge}\hfill {England}\\Summer Course Program\hfill {\em Aug - Sep 2017}\\- Learn Academic Writing and Speaking, Visual Arts and Architecture in Cambridge.
\end{rSection}

% \begin{rSection}{Research Profile}
%  \item Student President, Technology and Research, IIT Gandhinagar
% \end{rSection}

\begin{rSection}{Publication}
 \item The 83rd National Convention for Information Processing Society of Japan(IPSJ)\hfill {\em Mar 2021}\\{\bf Solving The Multi-Depot Vehicle Routing Problem using Deep Reinforcement Learning}\\\hfill {Rintaro Sato, Yasutaka Tsuji}
\end{rSection}

\begin{rSection}{Scholarship}
 \item {\bf Tobitate Young Ambassador Scholarship}\hfill {\em Aug 2019}\\- Scholarship for the students to develop into globally minded professionals needed by society, especially industry, and who can succeed in a global world

\end{rSection}

\begin{rSection}{Certification}
 \item Databases and SQL for Data Science on Coursera\hfill {\em Aug 2020}
\item Accelerated Computer Science Fundamentals Specialization on Coursera\hfill {\em Apr 2020}
\item Convolutional Neural Networks on Coursera\hfill {\em Aug 2019}
\item Sequence Models on Coursera\hfill {\em Aug 2019}
\item Neural Networks and Deep Learning on Coursera\hfill{\em Aug 2019}
\item Advanced Open Water Diver on PADI\hfill{\em Sep 2016}
\end{rSection}

\end{document}
